\emph{This is a nice study that proposes an information-theoretic
rationale for weighting ESM outputs when computing multi-model average
projections. The approach constructs weights from the divergence between
each ESM's output distribution and the observed-climate distribution,
thereby rewarding models that align more closely with an observational
product. The method is demonstrated on an ensemble of eight CMIP6 models
to project net ecosystem exchange of CO\textsubscript{2} and net biome
production, with weighting schemes calibrated against observational
datasets. I found the study well written, with a clear and intuitive
presentation of the information-theoretic background. These concepts are
often missing from discussions of climate-model post-processing, and it
is refreshing to see them used here. I also enjoyed learning about the
connection between cross-entropy and AIC. I have a small quibble with
calling the KL divergence a distance, but I will not press the point
because the term likely helps build intuition.}

We thank the reviewer for the positive and constructive evaluation of
our article. We agree with the reviewer's concern about our use of the
term `distance', and acknowledge that `divergence' is a more appropriate
term. Most introductory textbooks on information theory make this
clarification, but we used the word in our original version assuming no
previous knowledge of readers on the details of distance metrics (norms)
in mathematics. However, we see now the potential source of confusion
given that a portion of the readers of this article might be familiar
with the mathematical definition of metric and norm.\\
To address this issue, we added in section 2 a paragraph clarifying the
difference between `distance' and `divergence', and point out that for
the purpose of this article, with treat both terms as synonyms.

\emph{Although I am not deeply familiar with all work on combining ESM
outputs, my understanding is that another common strategy is to reward
models that (i) simulate today's climate well and (ii) remain close to
the ensemble consensus for future change. The manuscript cites earlier
work (e.g.~Tebaldi \& Knutti) at several points, but a fuller discussion
of how existing methods compare would be valuable. Readers will want
guidance on when this weighting scheme should be preferred and why.}

The literature on multi-model ensamble averages is relatively rich, and
there are more approaches than the one mentioned by the reviewer here.
It is not our intention to provide here a literature review on this
topic, as other reviews already exist. Nevertheless, we added a
paragraph in the Discussion section in which we briefly mention the type
of available approaches as reviewed by Tebaldi \& Knutti (2007), with
additions of more recent and relevant references.

\emph{In addition, I have a minor comments/questions that I hope the
authors will be able to address before this is considered for
publication.} \emph{L95-100 : could the authors expand on why the
approximation dismissing K is appropriate? I know this is discussed
later in the manuscript as a limitation of the proposed method, but I
think it would be useful to also have an argument at this point on why
that's a reasonable approximation to start with.}

We modified this section by first providing a version of equation 5 that
includes \(K\), followed by the approximation version without \(K\). We
explain the reason for not including \(K\) here following the same
arguments provided in section 5.

\emph{L105-110 : ``but given the absence of any other method for
obtaining a log-likelihood function of a parameterized ESM with respect
to data'' I would recommend nuancing this statement. There exists
methods out there that allow to model loglikelihood functions
(e.g.~variational approaches). This doesn't diminish the proposed
approach, since it might be the simplest first step to take, and in the
Occam's razor philosophy, it makes sense being explored and worthy of a
publication.}

We modified this paragraph based on the reviewer's suggestion. It is
true that models that use some parameterization schemes such as the
4D-var method and its variants, provide the possibility to obtain a
likelihood function. However, these approaches are often used in one
component of the model, and not necessarily to parameterize a
fully-coupled ESM. Nevertheless, it may be possible to use results from
these optimization approaches to add some information on the likeliihood
function for some component of the ESM.

\emph{Eq 13 : Am I correct in saying that the weights end up being
\(w_i = 1/\sigma_i / \sum 1/\sigma_i\)? I think it would be useful to
explicitly include this in the manuscript. The current presentation aims
for a greater level of generality in its formalism, which is
commendable, and could apply to any choice of distance metric A.
However, for the particular choice made by the authors here, the
expression of wi simplifies a lot and becomes very interpretable : we
simply give more weight to model that have better least square agreement
with the observational product.}

We thank the reviewer for this important comment. Indeed, for our
particular choice of the metric \emph{A}, the weights can also be
expressed as the inverse of the variance between model output and
observations. In fact, in the literature on maximum likelihood
estimation, inverse-variance weighting emerges as an efficient estimate
of the mean for populations in which the variances are known and the
mean is unknown. This is an interesting connection between the
information-theoretic approach and the maximum-likelihood theory, which
converges to the formulas of inverse-variance weighting for our choice
of metric \emph{A}. Based on this result, we modified some of the
presentation of the theoretical results, making a better link to the
maximum likelihood theory, showing alternative formulas for the weights
based on the inverse-variance equations, and presenting a simpler
formula for the overall variance. In addition, we found that
inverse-variance weighting methods are common in meta-analyses and in
the biomedical literature, so we added a few sentences in the Discussion
showing these alternative use of the method.

\emph{Eq 15 : Is this supposed to be a definition of the uncertainty or
the variance of \(\bar{x}\)? If the latter, I don't understand how it is
derived, if the former I would suggest not using \(\bar{x}\) as a
subscript.}

We modified and clarified the representation of variance and
uncertainty. Given the previous result on the weights being identical to
inverse-variance weighting, we used this result to provide a formal
definition of variance. To represent uncertainties, we changed the
equations to express them as predictions intervals given that this is a
more appropriate way to express the uncertainties for predictions
outside the time range where model output and observations overlap.

\emph{With these points addressed, I believe the paper will make a
valuable contribution and be ready for publication.}

\emph{Just out of curiosity : I appreciate that the distance A is only
interpretable as a relative metric. I've nonetheless always been curious
about its interpretation in ``informational units''. What I mean is that
Shanon entropy measure information in bits, which can be argued to be an
intepretable unit. I guess this doesn't translate immediately here since
in the continuous setting we're using the differential entropy which is
homogeneous to x. But have you thought of ways to make its values as an
absolute metric interpretable?}

The original article of Kullback \& Liebler (1951) helps to arrive at an
interpretation of our proposed metric \emph{A}, and the model
differences \(\Delta\). It is important to keep in mind that the origin
of the KL divergence was in the context of the statistical problem of
discriminating between two populations. Therefore, the interpretation of
the KL distance in this context is of the information available to
distinguish between two statistical populations (Kullback \& Libler
1951, pg. 80). More generally, one can also interpret KL divergence as
the information available to distinguish between two probability
distributions. This information would be measured in bits or nats
depending on the base of the logarithm. Regarding our definition of
\emph{A} and its interpretation. \emph{A} is based on the log-likelihood
between model predictions and observations to approximate the KL
divergence, so we can interpret \emph{A} as the available information to
discriminate between the distribution of the model and the observational
product. We used the base \(e\) logarithm, so \emph{A} is measured in
nats. Now, the values of \(\Delta_i\) are differences among the values
\(A_i\) for individual models and the values of \(A\) of the model with
the lowest divergence to observations. Therefore, we can interpret the
values \(\Delta_i\) as the information available to discriminate between
each individual model and the value of the model with the lowest
divergence with the observations, measured in units of nats.

\textbf{References}

Kullback, S. and Leibler, R. A.: On Information and Sufficiency, The
Annals of Mathematical Statistics, 22, 79--86,
http://www.jstor.org/stable/2236703, 1951.

Tebaldi, C. and Knutti, R.: The use of the multi-model ensemble in
probabilistic climate projections, Philosophical Transactions of the
Royal Society A: Mathematical, Physical and Engineering Sciences, 365,
2053--2075, https://doi.org/10.1098/rsta.2007.2076, 2007.
