\emph{Overall, the manuscript provides a solution for an important task
-- averaging of ESM ensembles - that is both well-grounded in theory and
easy to apply. The (sometimes strong) assumptions are clearly stated,
and the method overall is well presented in terms of derivation and
examples.}

\emph{A few questions remain: 1) Calculation of models weights. In eqs
(7) -- (15), I take it that index ``i'' goes over all models except the
best performing model (``m''), which serves as the reference. Does that
mean the best performing model will not be included in the weighted
average? Please clarify.}

The index \emph{i} runs over all models, including the best performing
model \emph{m}. For the calculation of the weights, the best performing
model has a value \(\Delta_m =0\), which implies that the value of the
numerator in the computation of the weights (eq. 13) is equal to 1, and
for all other models the value in the numerator is less than 1.

\emph{2) Patterns of model performance in space and time (see also line
180): The authors resolve spatial patterns of model performance by
calculating model weights grid-by-grid over all points in time. This
inherently assumes temporal invariance of relative model performance,
which clearly is not the case. Therefore I wonder if it would not be
more appropriate to derive model residuals (and deltas) from space-time
regions rather than time-regions alone. Please comment.}

In principle we agree in that model performance should account for
spatial and temporal covariations. However, it is not trivial to include
these covariations in our information- theoretic approach because they
have to be treated in the context of mutual information. The same
reasoning applies for covariation among different models, which should
be treated as mutual information and not just quantifying a covariance
matrix. We believe that this is a topic that deserves further
investigation, and we added a paragraph in the Discussion section
addressing this topic.

\emph{3) In Sect. 4, simple averaging is used as a benchmark to compare
the weighted averaging proposed by the authors. While e.g.~Fig. 5
clearly show differences among the methods, it is not clear if the
weighted average really provides the better (in terms of smaller
disagreement from observations) estimate that the simple average. I am
quite sure this will be the case, but please add and discuss the related
numbers.}

From the theoretical point of view, the inverse-variance weighted
average is the most efficient estimator of the mean under the Maximum
Likelihood estimation theory. The numerical results support this claim
showing a much reduced variance in comparison to the weighted average,
and narrower prediction intervals.

\emph{My overall recommendation is to publish after these minor points
have been suitably addressed.}
